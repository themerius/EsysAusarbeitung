\section{Beispiel Python}


Python, dessen Name sich von der Künstlergruppe Monty Python ableitet, ist um
1990 von dem Niederländer Guido von Rossum entwickelt worden. In dieser
Sprache wird auf die klare Sprachsyntax und gute Lesbarkeit sehr großen Wert
gelegt. Geschaffen wurde die Sprache als Brücke zwischen Shell-Skripten und
C-Programmen.

Es handelt sich dabei um eine interpretierte, interaktive, Objekt-orientierte
und funktionale Sprache, die aber auch ein einfaches Skript sein kann. Es gibt
klar definierte Konzepte, wie Module für Namenräume, dynamische Typesierung
und simple, zugleich mächtige High-Level-Datenstrukturen wie Listen und
Verzeichnisse (Dictionaries.) Zudem ist Python sehr leicht via C/C++
erweiterbar, somit können Wrapper um C-Programme gebaut werden, so dass ein
erleichterter und komfortabler Zugriff via Python möglich wird. So können bei
Bedarf Programmzeilen in hart optimierten C-Code ausgelagert werden und Python
genutzt werden -- gerade im eingebetteten Kontext durchaus ein Vorteil.

Durch die aktive Entwicklergemeinde hat sich eine sehr große und mächtige
Standard-Bibliothek entwickelt.\cite{pyref-library} Was nicht zu
unterschätzen ist, gerade weil dadurch die schon sehr portablen
Python-Programme, auf diese Basis zurückgreifen können, die ohne Konfiguration
auf anderen Systemen vornehmen zu müssen.

Aber Python ist auch sehr leicht über externe Bibliotheken erweiterbar.
Zudem gibt es neben der großen Standard-Bibliothek sehr viele Pakete, die via
pipy-Index zur Verfügung stehen -- dieser Umfasst zur Zeit 21811
Pakete\footnote{Quelle: \url{http://pypi.python.org/pypi}, Zugriff am 22.6.12}.
Das Python-Ecosystem ist sehr umfassend!


\subsection{Architektur}


\subsubsection{Interpreter}


Die Python-Referenzimplementierung CPython ist in der Programmiersprache C
geschrieben. Ein Python-Skript wird über den CPython-Interpreter ausgeführt;
der Interpreter stellt alle Funktionalität bereit um das Skript “just in time”
auszuführen.

Module sind einfache *.py Dateien, die mit dem Schlüsselwort import als
Namensraum importiert werden können. Es kann dann auf Klassen und Funktionen
zugegriffen werden, die sich innerhalb der Datei befinden. Pakete sind
Ordner die eine \_\_init\_\_.py und auch wieder Module enthalten. Die Pakete
bzw. Module werden vom Interpreter relativ zu seinem Ort wo er von z.B. einer
Shell aufgerufen wurde aufgespürt. Der Interpreter spürt aber Pakete/Module
auch über die Umgebungsvariable PYTHONPATH auf oder über fest definierte
Ordner wo sich systemweite Pakete, wie die Standardbibliothek befinden.
Der Interpreter unterstützt eine selbständige Speicherverwaltung; eine
ausführliche Erklärung findet sich in [Kapitel x].

In CPython gibt es den General Interpreter Lock (GIL), welcher die CPython
Implementierung vereinfacht, indem er nur einen Thread zur gleichen Zeit
Python-Bytecode ausführen lässt. Threading ist durchaus möglich, um z.B.
Nicht-Blockierende Programmabschnitte zu ermöglichen -- aber mehrere CPU-Kerne
bringen der Anwendung nicht mehr Performanz.\cite{pyref-reference}


\subsubsection{Syntax}


Um ein Gefühl zu bekommen, wie Python-Code aussieht, ist hier ein Code-Beispiel
gezeigt, indem die meist verwandten Konstruktionen aufgezeigt werden.


\lstinputlisting[language=Python]{code/pycharm.py}


Es gibt noch eine Besonderheit bei den Python-Objekten: Dort muss explizit der
self-Zeiger bei jeder Funktion explizit mit angegeben werden. Was der Python
Philosophie  “Explicit is better than implicit”\footnote{Quelle: \url{http://www.python.org/dev/peps/pep-0020/}} entspricht.


\subsubsection{Dokumentation}


Die Dokumentation in Python erfolgt über so genannte Docstrings. Diese
Docstrings werden an die Funktions-, Klassenrümpfe bzw. Modulrümpfe gehängt
(siehe Syntax-Beispiel.) Die Docstrings können während der Laufzeit mit
`funktionsname.\_\_doc\_\_` ausgelesen werden. Mit dem beliebten
Dokumentationsgenerator `sphinx` kann aus diesen Docstrings automatisch
eine Dokumentation generiert werden.
Beispiele, wie die Funktion benutzt wird, werden gerne als Doctests innerhalb
der Docstrings angegeben.


\subsection{Garbage Collection}


Garbage Collection (GC) ist ein Prozess, der nicht mehr benötigter Speicher
automatisch freigibt, so dass sich der Programmierer nicht mehr explizit
darum kümmern muss. Python macht intern in der unteren C-Ebene starken
Gebrauch von mallock() und free(), daher braucht Python eine Strategie, um
Speicherlecks zu vermeiden und verwendet dazu (a) Referenzzählen und (b)
optional einen GC der Referenzzyklen aufspüren und brechen kann. (\cite{pyref-reference}, Seite 81)

Ergänzung: Typen die keine Referenzen auf andere Objekte halten, oder
lediglich Referenzen auf atomare Typen haben, müssen keine explizite GC
unterstützen. (\cite{pyref-c-api}, S. 138)

Python unterscheidet zwischen (a) Referenzzählen und (b) der automatischen
optionalen GC (\cite{pyref-extending}, S. 13):


\begin{enumerate}[(a)]

  \item Referenzzählen

        Jedes Python-Objekt hält eine Variable, worin die Anzahl der
        Referenzen gezählt wird, die von anderen Objekten in Besitz sind.
        Wenn eine Referenz auf ein Objekt in Besitz genommen werden will,
        muss es die Py\_INCREF-Methode (Variable hochzählen) des Objekts
        aufrufen werden. Wenn der Besitzer die Referenz verwerfen will, so
        wird Py\_DECREF (Variable runterzählen) aufgerufen. Sobald die
        Zählvariable bei Null angekommen ist, wird das Objekt zerstört.
        Dieses Verhalten ist in der untersten Ebene bzw. C-Ebene der Sprache
        verdrahtet -- zumindest gilt das für die offizielle
        CPython-Referenzimplementierung.

        Es gibt auch die Möglichkeit, dass Referenzen ausgeliehen werden
        können. Wenn das der Fall ist, wird kein Py\_DECREF vom Leihenden
        aufgerufen (die Referenz kann vom Leihenden also nicht zerstört
        werden.) Vorteil: Der Leihende muss nicht die Verantwortung über die
        Zerstörung der Referenz tragen; Nachteil: Der Leihende läuft Gefahr
        schon freigeräumten Speicher zu verwenden.

  \item Automatische GC

        Das was unter Python als automatischer GC gilt, ist optional und
        kann zur Lauftzeit deaktiviert werden. Dazu gibt es ein Interface,
        welches über das `gc`-Modul bereitgestellt wird, wo noch mehr
        Parameter eingestellt werden können, um das GC-Verhalten
        feinabzustimmen.

        Die automatische GC bietet eine verzögerte Erkennung von zyklisch
        zusammenhängenden Garbage-Objekten, eben diese die nicht via
        Referenzzählen aufgespürt werden können; z.B. wenn ein Objekt auf
        sich selbst eine Referenz hält.

        Die automatische GC ist also nur eine Ergänzung zum Referenzzählen.

\end{enumerate}


Der automatische, aber optionale, GC sorgt für ein nicht-deterministisches
Verhalten (durch die verzögerte Erkennung der Zyklen, die irgendwann
auftreten kann), was gerade für harte Echtzeitbedingungen zu schwer zu
prognostizierender Worst-Case-Analyse führt. Wenn man den GC deaktiviert,
muss jedoch darauf geachtet werden, dass keine Referenz-Zyklen auftreten,
was ein potentielles Speicherleck hervorrufen kann. (\cite{pyref-library}, S. 1084)

Das Referenzzählen kann in eine Worst-Case-Analyse besser aufgenommen werden,
da sich das Problem ebenfalls auf der C-Ebene abspielen würde -- und es gibt
durchaus realzeitfähige Dynamic Storage Allocator Algorithmen, die wiederum
mehr vom Betriebssystem bzw. den verwendeten C-Bibliotheken abhängen. Dieses
Verhalten sollte und kann auch in die Worst-Case-Berechnungen mit einfließen.\cite{malloc}

Referenzzählen ist also durchaus für den harten Echtzeiteinsatz geeignet.
Zumal eine Art von GC auch auf eingebetteten Systemen nicht verkehrt ist, da
dort Speicherlecks desaströs sind in Anbetracht der geringen
Speicherressourcen bzw. der einfachen oder nicht vorhandenen
Speicherverwaltung des eingebetteten Betriebssystems.\cite{refcount}

Wenn beim Programmstart lediglich statischer Speicher alloziert wird, so
sollte die GC sowieso eine recht untergeordnete Rolle spielen.


\subsection{Setup-Szenarien}


\subsubsection{Embedded Linux}


Linux ist ein weit verbreiteter Open-Source Betriebssystem-Kernel. Der Kernel
ist sehr flexibel in der Konfiguration, soll heißen, er kann vom Supercomputer
bis zum kleinen eingebetteten System verwendet werden. Mindestens wird jedoch
ein 32-Bit Prozessor vorausgesetzt.

Der Standardkernel kann lediglich weiche Realzeit garantieren, aber es gibt
Projekte die versuchen den Kernel auch fit für harte Realzeit im eingebetteten
Kontext zu machen. Beispiele hierfür sind RTLinux, µClinux oder avr32linux.
Realtime Linux ist eine Abzweigung des Standardkernel, welcher eine
verbesserte Unterbrechbarkeit und hochauflösende Timer bietet; es fließen hin
und wieder diese Modifikationen auch in den Standardkernel ein. (\cite{dipl}, S. 5)

Im folgenden werden verschiedene Python Implementierungen für (embedded) Linux
vorgestellt:


\begin{itemize}

  \item CPython

        Auf die Implementierung von CPython sind wir schon im Kapitel
        (Architektur, GC) ausführlich eingegangen.

        CPython muss gegen die glibc oder uClibc gebaut werden. Selbst wenn
        viele nicht benötigte Pakete aus der Standardbibliothek entfernt
        werden, benötigt der Interpreter noch immer mindestens 725 KB.
        (\cite{embeddedlinux}, Kap. 4.6)

  \item tinypy

        Ist eine minimalistische Python-Implementierung, die nur 64 KB braucht.\cite{tinypy}
  \item IronPython

        IronPython ist eine open-source Implementierung von Python für das
        .NET Framework. Die in C\# geschriebene Umgebung kann sowohl .NET-
        als auch Python-Bibliotheken verwenden und macht es den
        .NET-Programmiersprachen auf einfache Weise möglich Python-Code
        auszuführen.

        IronPython lässt sich auf einem (embedded) Linux mittels der Mono
        Virtual Machine ausführen und bietet gute Möglichkeiten zum Debuggen.
        Der Mono Debugger unterstützt Remote-Debugging, wofür nur ein kleiner
        Debug-Server auf dem eingebetteten System notwendig ist, und bietet so
        eine Übersicht der Threads inklusive Stacktraces und Inhalten der
        Variablen, zudem lassen sich Breakpoints festlegen und die Anweisungen
        können schrittweise durchsprungen werden (\cite{ironpython}, Kap. 3).

        Der von IronPython generierte CIL-Bytecode kann als .NET-Assembly
        gespeichert werden, was ein erneutes Ausführen durch die Virtual
        Machine um einiges beschleunigt.

\end{itemize}


Im Zuge der Diplomarbeit von Volker Thoms \cite{dipl} sind eine Reihe von
Python-Modulen für den komfortablen Gerätezugriff in Verbindung mit einem
embedded Linux entstanden bzw. in Zusammenhang gebracht. Es handelt sich dabei
um Wrapper für die Unix-Geräte-Dateien (/dev/*):


\begin{itemize}

  \item socket: Teil der Standardbibliothek
  \item pyserial: Für serielle Schnittstellen, ein Wrapper um termios.
  \item py-smbus: Wrapper um /dev/i2c-x
  \item py-spi: Wrapper um /dev/spidevX.Y. (X=bus, Y=client)
  \item py-pwm: Generische Pulsweiten Modulations API, die den Hardwarekanal ansprechen kann
  \item py-softpwm: Software PWM als Kernelmodul, Wrapper um /dev/softpwm \cite{dipl}

\end{itemize}


PyET ist eine Sammlung von Programmen, Modulen und Skripten, welche die
Entwicklung von eingebetteten Systemen mit Python vereinfacht. Es existieren
beispielsweise Pakete für “Background Debug Mode for Motorola processors”
(PyBDM) und “Boundary Scan for IEEE-1149.1 (JTAG) devices” (PyBSC), welche
von dem PyET-Projekt entwickelt wurden.\cite{pyet}


\subsubsection{python-on-a-chip}


Einen sehr spannenden und gänzlich anderen Ansatz verfolgt das
python-on-a-chip-Projekt. Das Projekt entwickelt eine reduzierte Python
virtuelle Maschine, die PyMite VM, zugeschnitten auf 8-Bit und größere
Mikrocontroller. Es kommt gänzlich ohne Betriebssystem aus und kann somit
direkt auf dem Mikrocontroller arbeiten, mit gleichzeitig geringem
Resourcenverbrauch von ca. 64 KB Flash-Speicher und mindestens
4 KB Arbeitsspeicher.

Jedoch gilt zu beachten, dass es sich hierbei um ein reduziertes Python
handelt und somit nicht alle Python-Tricks umgesetzt werden können. Es kann
somit auch nicht auf die normale Python-Standard-Bibliothek gesetzt werden
bzw. andere normale Bibliotheken, die das vollständige Python voraussetzen.
Der Entwickler muss also auf die Treiber, Bibliotheken und Tools setzen, die
das python-on-a-chip-Projekt liefert.

Die VM kann mehrere Threads mit dem Round-Robin Scheduling-Verfahren laufen
lassen. Als GC-Methode kommt ein Mark-and-Sweep-Algorithmus zum Einsatz. \cite{p14p}

Die Worst-Case-Ausführungszeit von Mark-and-Sweep entspricht der quadrierten
dynamischen Speichergröße (Heap).\cite{sweep} Und sobald eine vernünftige
Worst-Case-Analyse möglich ist, kann auch harter Realzeit genüge getan werden
-- wenn auch die Analyse mit etwas mehr Aufwand verbunden ist als mit
manueller Speicherverwaltung.
