\section{Fazit}


Zunächst wurde ein Überblick verschafft, was interpretierte Sprachen
“Skriptsprachen” ausmacht. Der große Vorteil sind die agile Entwicklung und
Nachteil ist die Performanz bzw. das nicht-deterministische Verhalten, welches
gerade durch den Gargabe Collector ausgelöst wird.

Im Konkreten Beispiel Python sind die Implementeriungsdetails der Sprache zur
Sprache gekommen, insbesondere die Details der Referenzimplementierung CPython
und ob es möglich ist, diese Sprache harten Realzeitanforderungen auszusetzen,
ohne all zu viel von der agilen und komfortablen Entwicklung zu verlieren.
Das hängt jedoch von zwei Faktoren ab: (1) dem eingesetzen Betriebssystem und (2) ob
eine vernünftige Worst-Case-Analyse der Spracheigenschaften bzw.
Sprachimplementierung, aber insbesondere des eingesetzten Garbage Collection
Verfahrens, möglich ist. Es konnte gezeigt werden, dass es durchaus Garbage Collection
Verfahren, wie das Referenzzählen, gibt, die auch für harte Echtzeit geeigenet
sein können und CPython setzt ein solches Verfahren ein. Wenn ein embedded Linux als
Betriebssystem eingesetzt wird, muss jedoch etwas Aufwand getrieben werden,
dieses für harte Echtzeit zu rüsten.

Aber es gibt nicht nur die CPython-Implementeriung sondern auch eine Reihe
anderer Python-Interpreter die es sich zum Ziel gesetzt haben minimalere
Interpreter zu entwickeln, die insbesondere einen geringen Resourcenverbrauch
aufweisen.

Das python-on-a-chip Projekt sticht heraus, da es den Python Interpreter
direkt auf einem Mikrokontroller lauffähig macht; ganz ohne Betriebssystem.
Zudem ist die vom Projekt eingesetze Garbage Collection
Worst-Case-analysierbar und somit kann harte Realzeit garantiert werden.

Was aber bei allen Skriptingsprachen gilt ist, dass die Worst-Case-Analyse
zur Bestimmung der harten Realzeitanforderungen komplexer ist und einige
Fallen mehr existieren.

Jedoch ist Fakt, dass eine einfachere und schnellere Entwicklung möglich ist,
die für weiche Realzeitanforderungen mehr als genügt, sofern genug
Hardware\-ressourcen zur Verfügung stehen.


\newpage
\section{Fragen}


\begin{enumerate}

  \item Was ist ein Hauptvorteil und ein Hauptnachteil von interpretierten Sprachen?

        Vorteil: agile Entwicklung, Nachteil: relativ viel Overhead

  \item Ein Vor- aber auch Nachteil ist die Garbage Collection. Warum kann GC zu einem Nachteil werden (bei harten Realzeitanforderungen)?

        Je nach eingeseten Algorithmus: Gänzlich nicht-deterministisches
        Verhalten oder es ist schwieriger eine Worst-Case-Analyse zu erstellen.

  \item Was zeichnet Python besonders für den eingebetteten Einsatz aus?

        Da es als Brücke zwsichen Shell-Skripten und C-Programmen konzipiert
        ist, ist es relativ leicht, wenn nötig, via C/C++ zu erweitern.

\end{enumerate}
