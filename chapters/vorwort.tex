\section{Vorwort}

Eingebettete Systeme sind immer leistungsfähiger geworden, mittlerweile gibt
es sehr günstige 32-Bit Mikrocontroller-Architekturen mit relativ hohen
Taktfrequenzen und dennoch geringem Energieverbrauch. Dadurch ergibt sich die
Möglichkeit größere Betriebssysteme, wie Linux, einzusetzen um somit weniger
hardwarenah bzw. abstrakter programmieren zu können. Jedoch ist dies für
gewöhnlich mit etwas mehr Overhead bzw. erhöhtem Rechenaufwand verbunden,
aber zugleich eleganter und schneller zu entwickeln.

Gerade interpretierte Programmiersprachen sind dazu geschaffen, um schnell
eleganten Programmcode zu schreiben, aber meistens leidet darunter die
Aus\-führ\-ungs\-gesch\-windig\-keit aufgrund von diversen “High-Level”-Funktionen. In
dieser Ausarbeitung soll geklärt werden ob solche Programmiersprachen,
insbesondere die Programmiersprache Python, auch zur Realzeitfähigkeit taugen.
Dazu werden deren Vor- und Nachteile im Realzeit-Kontext betrachtet und
spezielle Problematiken aufgezeigt.


Python ist eine sehr beliebte Sprache geworden und wird mittlerweile auf
nahezu jeder Standard-UNIX-Distribution mit ausgeliefert und bietet sehr
viele flexible Möglichkeiten, sowie eine große Standard-Bibliothek und
Erweiterbarkeit via C/C++. Aus diesem Grund bietet sich Python als eine
attraktive Sprache für den eingebetteten Bereich an, weshalb in dieser Arbeit
Python als näher beleuchtetes Beispiel herangezogen wird. Der zentrale Punkt
der Diskussion ist die Garbage Collection und welche Rolle diese gerade für
harte Echtzeit spielt.
Interessant sind hier aber auch die verschiedenen Möglichkeiten, Python auf
einem eingebetten System auszuführen.
