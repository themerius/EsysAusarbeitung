\section{Interpretierte Sprachen und Realzeit}


\subsection{Definitionen}


Interpretierte Sprachen sind Programmiersprachen, deren Quellcode zur Laufzeit
durch einen Interpreter eingelesen, analysiert und ausgeführt wird. Dadurch
ermöglichen sie ein erleichtertes Programmieren, da sie viel Komplexität
verstecken und einen höheren Abstraktionsgrad ermöglichen -- so müssen zum
Beispiel Zeiger nicht mehr explizit verwaltet werden, oder Objekte werden von
einem Garbage Collector automatisch aufgeräumt.


Realzeit bedeutet, dass das System innerhalb definierter Zeitspannen zu
reagieren hat. Das bedeutet also, dass insbesondere die Software innerhalb
dieser Zeitspanne ihre Berechnungen erledigt haben muss.
Wenn ein System oder Prozess weiche Realzeitanforderungen hat, kann auch mal
eine Frist, also die Reaktion war außerhalb der Zeitspanne, verpasst werden.
Hat ein System allerdings harte Echtzeitanforderungen, so darf es um keinen
Preis die Frist verpassen, da sonst die Folgen fatal wären. Beispielsweise
ein Airbag, der erst nach dem Unfall auslöst, wäre unbrauchbar.


\subsection{Vorteile von interpretierten Sprachen}


Hier sind die Vorteile aufgelistet, die durch interpretierte Sprachen oder
Skriptingsprachen entstehen und gleichzeitig wird ein Ausblick gegeben, was
dies für den eingebetteten Kontext bzw. Realzeitbetrieb bedeutet.


\begin{itemize}

  \item \textbf{erleichterte Programmierung}

        Was interpretierte Sprachen auszeichnet ist ihre Wendigkeit und die
        daraus resultierende kürzere Entwicklungszeit der Software. Somit kann
        Software schneller und einfacher umgebaut werden, aber auch leichter
        getestet werden. Daher sind interpretierte Sprachen wunderbar für
        rapide Entwicklung geeignet, weshalb diese Sprachen gerade im Bereich
        Web sehr verbreitet sind.

        Einer der Gründe ist die dynamische Typisierung, die die meisten
        interpretierten Sprachen einsetzen: Es muss nicht mehr über alle Typen
        nachgedacht werden bzw. man fällt nicht mehr so leicht in diverse
        Fallen, die durch statische Typisierung entstehen. Ebenso ist es
        jederzeit möglich, die logische Programm-Struktur zu ändern, da diese
        erst zur Laufzeit zusammengestellt wird; es lassen sich Variablen,
        Klassen und Funktionen hinzufügen. (Dadurch entsteht aber auch ein
        gewisser Overhead, wodurch diese Sprachen deutlich langsamer sein
        können.)

        Es gibt viele komfortable und durch die Sprache vordefinierte (u.U.
        auch sehr mächtige) Konstrukte, die es ermöglichen eleganten Code zu
        schreiben, welcher kompakt und ausdrucksstark ist. Die Sprache hat
        somit ihre eigenen Paradigmen, welche eine hohe Abstraktion bieten.
        Dank solcher Abstraktionen, muss sich der Programmierer kaum  noch
        Gedanken über Zeiger und ihre Verwaltung machen, das übernimmt alles
        der Interpreter bzw. die virtuelle Maschine der Sprache.

        Kein Compilieren, kein Binden, interaktives Debugging.

        Der Programmierer arbeitet also mehr problemorientiert, als an den
        Problemen mit der Programmiersprache zu kämpfen.

  \item \textbf{erleichterte Portierbarkeit}

        Interpretierte Sprachen lassen sich relativ leicht auf neue
        Architekturen portieren, denn lediglich der Interpreter muss angepasst
        werden, der Quellcode der Programme bleibt unberührt.

  \item \textbf{Garbage Collection}

        Die Garbage Collection ist eine der High-Level Abstraktionen, die vom
        Interpreter bzw. der virtuellen Maschine der Programmiersprache
        angeboten werden. Wenn eine GC angeboten wird, bedeutet das für den
        Programmierer, dass er sich nicht mehr um die Speicherallokierung und
        Deallokierung kümmern muss. Das minimiert potentielle Fehlerquellen
        und erhöht die Entwicklungsgeschwindigkeit → mehr Fokus auf das
        Problem, welches gelöst werden will.

\end{itemize}


\subsection{Nachteile von interpretierten Sprachen}


Hier sind die Nachteile aufgelistet, die durch interpretierte Sprachen oder
Skriptingsprachen entstehen und gleichzeitig wird ein Ausblick gegeben, was
dies für den eingebetteten Kontext bzw. Realzeitbetrieb bedeutet.


\begin{itemize}

  \item \textbf{Overhead}

        Es kann eine langsamere Ausführungszeit bei interpretierten Sprachen
        entstehen durch das höhere Abstraktionslevel und Vereinfachungen.
        Der Programmcode muss zur Laufzeit interpretiert bzw. compiliert
        werden; in regelmäßigen Abständen wird eine Garbage Collection
        vorgenommen, wodurch der eigentliche Programmablauf unterbrochen oder
        verlangsamt wird.

        Abgesehen von dem zusätzlichen Rechenaufwand gibt es einen Mehrbedarf
        an Arbeitsspeicher, zum einen benötigt der Interpreter (bzw. die
        virtuelle Maschine) zusätzlichen RAM für seinen Programmcode und seine
        Daten, zum anderen braucht der interpretierte Code ebenfalls mehr
        Speicher als beispielsweise C-Code.

        Die Ausführung mittels des Interpreters führt zu einer indirekteren
        Systeminteraktion, die Zugriffe auf das Betriebssystem (“Syscalls”)
        sind mit Overhead verbunden, da die Schnittstellen abstrahiert sind
        und der Interpreter die Aufrufe weiterleiten muss.

  \item \textbf{dynamische Typisierung}

        Die dynamische Typisierung kann allerdings auch ein Nachteil sein,
        da der Typ der Variablen unbekannt ist, was gerade in selten
        durchlaufenen Programmteilen, die eventuell auch nicht richtig
        getestet wurden, zu unerwartetem Fehlern führen kann. Die dynamische
        Typüberprüfung kostet natürlich auch Rechenaufwand zur Laufzeit,
        zudem fallen Optimierungsmöglichkeiten wie das direkte Einfügen von
        Maschinencode statt eines Methoden- oder Funktionsaufrufs weg.

\end{itemize}


\subsection{Problematiken im Realzeit-Bereich}


So lange das System noch nicht an der Auslastungsgrenze ist (CPU-Rechenzeit),
sollte es kein Problem darstellen, wenn die Software etwas langsamer läuft.
Es kommt erst dann zu Problemen, wenn harte Echtzeit von Nöten ist, wenn der
Garbage Collector anspringt und das Programm somit ins Stocken gerät -- zumal
der GC nicht unbedingt zu deterministischen Zeiten anspringt, und auch nicht
bekannt ist, wie lange er aktiv ist. Es ist somit sehr schwierig ein
Wort-Case-Szenario zu erstellen und es müssen Möglichkeiten gesucht werden,
wie man diese Problematiken in den Griff bekommen kann, so dass auch harte
Realzeit realisiert werden kann.


Für Systeme die nur eine weiche Realzeit erfordern, sollten diese
Beschränkungen eher weniger das Problem sein. Was z.B. bei diversen 
"`echtzeit"' Web-Anwendungen der Fall ist.
