\begin{titlepage}
	\vspace*{7cm}
	\begin{center}
		\Huge
		Realzeitanforderungen an eine Scriptingssprache am Beispiel von Python\\
		\vspace{1cm}
		\large
		\textbf{Fakultät Informatik (TIB, SEB)}\\
		\today\\
		\vspace{2cm}
		Bumiller, Marc \emph{mabumill@htwg-konstanz.de}\\
		Hodapp, Sven \emph{svhodapp@htwg-konstanz.de}\\
	\end{center}
	\normalsize
	\vfill
	\textbf{Hochschule für Technik, Wirtschaft und Gestaltung Konstanz.} Betreuung durch Herrn Prof. Dr. Michael Mächtel (HTWG.) 

Es wird aufgeklärt, welche verschiedenen Anforderungen an interpretierte Programmiersprachen ("Scriptingsprachen") an ein realzeitfähiges bzw. eingebettetes System entstehen. Am Beispiel von Python soll geklärt werden, ob dort all diese Anforderungen umgesetzt werden können -- insbesondere die Garbage Collection ist ein zentrales Thema. Es soll ein Überblick geschaffen werden, welche Vorteile und auch Nachteile durch die Nutzung entstehen können. (Python auf Embedded Linux, python-on-a-chip)

\end{titlepage}
