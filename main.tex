%
% Esys Ausarbeitung
% Marc Bumiller
% Sven Hodapp
%


% -----------
% 1. Präambel
% -----------


% Allgemeine Einstellungen
% ------------------------
\documentclass[
	pdftex,%              PDFTex verwenden da wir ausschliesslich ein PDF erzeugen.
	a4paper,%             Wir verwenden A4 Papier.
	oneside,%             Einseitiger Druck.
	12pt,%                Grosse Schrift, besser geeignet für A4.
	halfparskip,%         Halbe Zeile Abstand zwischen Absätzen.
	%chapterprefix,%       Kapitel mit 'Kapitel' anschreiben.
	headsepline,%         Linie nach Kopfzeile.
	footsepline,%         Linie vor Fusszeile.
	bibtotocnumbered,%    Literaturverzeichnis im Inhaltsverzeichnis nummeriert einfügen.
	idxtotoc%             Index ins Inhaltsverzeichnis einfügen.
]{article}

\usepackage[utf8]{inputenc}
\usepackage[german]{babel}   % deutsche Silbentrennung
\selectlanguage{german}   % damit Table Of Contents Inhaltsverzeichnis genannt wird

\usepackage{geometry}   % Seitenränder einstellbar
\usepackage{textcomp}   % Sonderzeichen, wie Eurosymbol



% Bilder, Farben, farbige Tabellen
% --------------------------------
\usepackage{graphicx, color, colortbl}
\usepackage{array}       % Erweiterte Tabelleneigenschaften.
%\usepackage{floatflt}   % Bild kann von Text umflossen werden.



% Palatino Schrift
% ----------------
%\usepackage[T1]{fontenc}
%\usepackage[osf]{mathpazo}   % osf aktiviert Mediävalziffern/Minuskelziffern



% Sonstige Pakete
% ---------------
%\usepackage{anysize}   % Seitenränder verändern
%\usepackage{setspace}   % 1.5em Zeilenabstand \begin{onehalfspacing}
\usepackage{bibgerm}   % Anzeigestil des Literaturverzeichnis (gerabbrv)
\usepackage{listings}  \lstset{numbers=left, numberstyle=\tiny, numbersep=20pt} % Programmcode einfügen


% PDF Eigenschaften
% -----------------
\usepackage
[
	colorlinks=false,
	bookmarks = true,
	pdftitle={Praxissemester Bericht},
	pdfauthor={Sven Hodapp},
	pdfsubject={PSS},
	pdfkeywords={praxissemester, fraunhofer, htwg},
	urlcolor=blue,
	pdfstartview=FitH
]{hyperref}





% --------------------
% 2. Dokumenten Anfang
% --------------------

\begin{document}



% Deckblatt
% ---------
\begin{titlepage}
	\vspace*{7cm}
	\begin{center}
		\Huge
		Realzeitanforderungen an eine Scriptingssprache am Beispiel von Python\\
		\vspace{1cm}
		\large
		\textbf{Fakultät Informatik (TIB, SEB)}\\
		\today\\
		\vspace{2cm}
		Bumiller, Marc \emph{mabumill@htwg-konstanz.de}\\
		Hodapp, Sven \emph{svhodapp@htwg-konstanz.de}\\
	\end{center}
	\normalsize
	\vfill
	\textbf{Hochschule für Technik, Wirtschaft und Gestaltung Konstanz.} Betreuung durch Herrn Prof. Dr. Michael Mächtel (HTWG.) 

Es wird aufgeklärt, welche verschiedenen Anforderungen an interpretierte Programmiersprachen ("Scriptingsprachen") an ein realzeitfähiges bzw. eingebettetes System entstehen. Am Beispiel von Python soll geklärt werden, ob dort all diese Anforderungen umgesetzt werden können -- insbesondere die Garbage Collection ist ein zentrales Thema. Es soll ein Überblick geschaffen werden, welche Vorteile und auch Nachteile durch die Nutzung entstehen können. (Python auf Embedded Linux, python-on-a-chip)

\end{titlepage}




% Inhaltsverzeichnis anzeigen
% ---------------------------
\tableofcontents
\listoffigures
%\listoftables




% ---------
% 3. Inhalt
% ---------

\section{Vorwort}
Das Vorwort. \cite{abbr}



\addcontentsline{toc}{section}{Literaturverzeichnis}
\bibliographystyle{gerabbrv}
\bibliography{bib/references}

\end{document}


