%
% Esys Ausarbeitung
% Marc Bumiller
% Sven Hodapp
%


% -----------
% 1. Präambel
% -----------


% Allgemeine Einstellungen
% ------------------------
\documentclass[
	pdftex,%              PDFTex verwenden da wir ausschliesslich ein PDF erzeugen.
	a4paper,%             Wir verwenden A4 Papier.
	oneside,%             Einseitiger Druck.
	12pt,%                Grosse Schrift, besser geeignet für A4.
	halfparskip,%         Halbe Zeile Abstand zwischen Absätzen.
	%chapterprefix,%       Kapitel mit 'Kapitel' anschreiben.
	headsepline,%         Linie nach Kopfzeile.
	footsepline,%         Linie vor Fusszeile.
	bibtotocnumbered,%    Literaturverzeichnis im Inhaltsverzeichnis nummeriert einfügen.
	idxtotoc%             Index ins Inhaltsverzeichnis einfügen.
]{article}

\usepackage[utf8]{inputenc}
\usepackage[german]{babel}   % deutsche Silbentrennung
\selectlanguage{german}   % damit Table Of Contents Inhaltsverzeichnis genannt wird

\usepackage{geometry}   % Seitenränder einstellbar
\usepackage{textcomp}   % Sonderzeichen, wie Eurosymbol



% Bilder, Farben, farbige Tabellen
% --------------------------------
\usepackage{graphicx, color, colortbl}
\usepackage{array}       % Erweiterte Tabelleneigenschaften.
%\usepackage{floatflt}   % Bild kann von Text umflossen werden.



% Palatino Schrift
% ----------------
%\usepackage[T1]{fontenc}
%\usepackage[osf]{mathpazo}   % osf aktiviert Mediävalziffern/Minuskelziffern



% Syntax-Highlighting
% -------------------
\definecolor{darkgreen}{rgb}{0,0.5,0}  % needs usepackage color
\usepackage{listings}
\lstset{
	language=Python, 
	basicstyle=\footnotesize\ttfamily, 
	frame=lines, 
	numbers=left, 
	numberstyle=\tiny, 
	numbersep=5pt, 
	breaklines=true, 
	showstringspaces=false, 
	keywordstyle=\color{blue}, 
	commentstyle=\color{darkgreen}, 
	stringstyle=\color{red}
}



% Sonstige Pakete
% ---------------
%\usepackage{anysize}   % Seitenränder verändern
%\usepackage{setspace}   % 1.5em Zeilenabstand \begin{onehalfspacing}
\usepackage{bibgerm}   % Anzeigestil des Literaturverzeichnis (gerabbrv)
\usepackage{listings}  \lstset{numbers=left, numberstyle=\tiny, numbersep=20pt} % Programmcode einfügen


% PDF Eigenschaften
% -----------------
\usepackage
[
	colorlinks=false,
	bookmarks = true,
	pdftitle={Praxissemester Bericht},
	pdfauthor={Sven Hodapp},
	pdfsubject={PSS},
	pdfkeywords={praxissemester, fraunhofer, htwg},
	urlcolor=blue,
	pdfstartview=FitH
]{hyperref}





% --------------------
% 2. Dokumenten Anfang
% --------------------

\begin{document}



% Deckblatt
% ---------
\begin{titlepage}
	\vspace*{7cm}
	\begin{center}
		\Huge
		Python und eingebettete Systeme\\
		\vspace{1cm}
		\large
		\textbf{Fakultät Informatik (TIB, SEB)}\\
		\today\\
		\vspace{2cm}
		Bumiller, Marc \emph{mabumill@htwg-konstanz.de}\\
		Hodapp, Sven \emph{svhodapp@htwg-konstanz.de}\\
	\end{center}
	\normalsize
	\vfill
	\textbf{Hochschule für Technik, Wirtschaft und Gestaltung Konstanz.} Betreuung durch Herrn Prof. Dr. Michael Mächtel (HTWG.) 

Die verschiedenen Wege, wie die interpretierte Programmiersprache "python" auf eingebettete Systeme gebracht werden kann. Es soll ein Überblick geschaffen werden, welche Vorteile und auch Nachteile durch die Nutzung entstehen können. (Python auf Embedded Linux, python-on-a-chip)

\end{titlepage}




% Inhaltsverzeichnis anzeigen
% ---------------------------
\tableofcontents
\listoffigures
%\listoftables




% ---------
% 3. Inhalt
% ---------

\section{Vorwort}
Das Vorwort. \cite{abbr}


\section{Interpretierte Sprachen und Realzeit}


\subsection{Definitionen}


Interpretierte Sprachen sind Programmiersprachen, deren Quellcode zur Laufzeit
durch einen Interpreter eingelesen, analysiert und ausgeführt wird. Dadurch
ermöglichen sie ein erleichtertes Programmieren, da sie viel Komplexität
verstecken und einen höheren Abstraktionsgrad ermöglichen -- so müssen zum
Beispiel Zeiger nicht mehr explizit verwaltet werden, oder Objekte werden von
einem Garbage Collector automatisch aufgeräumt.


Realzeit bedeutet, dass das System innerhalb definierter Zeitspannen zu
reagieren hat. Das bedeutet also, dass insbesondere die Software innerhalb
dieser Zeitspanne ihre Berechnungen erledigt haben muss.
Wenn ein System oder Prozess weiche Realzeitanforderungen hat, kann auch mal
eine Frist, also die Reaktion war außerhalb der Zeitspanne, verpasst werden.
Hat ein System allerdings harte Echtzeitanforderungen, so darf es um keinen
Preis die Frist verpassen, da sonst die Folgen fatal wären. Beispielsweise
ein Airbag, der erst nach dem Unfall auslöst, wäre unbrauchbar.


\subsection{Vorteile von interpretierten Sprachen}


Hier sind die Vorteile aufgelistet, die durch interpretierte Sprachen oder
Skriptingsprachen entstehen und gleichzeitig wird ein Ausblick gegeben, was
dies für den eingebetteten Kontext bzw. Realzeitbetrieb bedeutet.


\begin{itemize}

  \item \textbf{erleichterte Programmierung}

        Was interpretierte Sprachen auszeichnet ist ihre Wendigkeit und die
        daraus resultierende kürzere Entwicklungszeit der Software. Somit kann
        Software schneller und einfacher umgebaut werden, aber auch leichter
        getestet werden. Daher sind interpretierte Sprachen wunderbar für
        rapide Entwicklung geeignet, weshalb diese Sprachen gerade im Bereich
        Web sehr verbreitet sind.

        Einer der Gründe ist die dynamische Typisierung, die die meisten
        interpretierten Sprachen einsetzen: Es muss nicht mehr über alle Typen
        nachgedacht werden bzw. man fällt nicht mehr so leicht in diverse
        Fallen, die durch statische Typisierung entstehen. Ebenso ist es
        jederzeit möglich, die logische Programm-Struktur zu ändern, da diese
        erst zur Laufzeit zusammengestellt wird; es lassen sich Variablen,
        Klassen und Funktionen hinzufügen. (Dadurch entsteht aber auch ein
        gewisser Overhead, wodurch diese Sprachen deutlich langsamer sein
        können.)

        Es gibt viele komfortable und durch die Sprache vordefinierte (u.U.
        auch sehr mächtige) Konstrukte, die es ermöglichen eleganten Code zu
        schreiben, welcher kompakt und ausdrucksstark ist. Die Sprache hat
        somit ihre eigenen Paradigmen, welche eine hohe Abstraktion bieten.
        Dank solcher Abstraktionen, muss sich der Programmierer kaum  noch
        Gedanken über Zeiger und ihre Verwaltung machen, das übernimmt alles
        der Interpreter bzw. die virtuelle Maschine der Sprache.

        Kein Compilieren, kein Binden, interaktives Debugging.

        Der Programmierer arbeitet also mehr problemorientiert, als an den
        Problemen mit der Programmiersprache zu kämpfen.

  \item \textbf{erleichterte Portierbarkeit}

        Interpretierte Sprachen lassen sich relativ leicht auf neue
        Architekturen portieren, denn lediglich der Interpreter muss angepasst
        werden, der Quellcode der Programme bleibt unberührt.

  \item \textbf{Garbage Collection}

        Die Garbage Collection ist eine der High-Level Abstraktionen, die vom
        Interpreter bzw. der virtuellen Maschine der Programmiersprache
        angeboten werden. Wenn eine GC angeboten wird, bedeutet das für den
        Programmierer, dass er sich nicht mehr um die Speicherallokierung und
        Deallokierung kümmern muss. Das minimiert potentielle Fehlerquellen
        und erhöht die Entwicklungsgeschwindigkeit → mehr Fokus auf das
        Problem, welches gelöst werden will.

\end{itemize}


\subsection{Nachteile von interpretierten Sprachen}


Hier sind die Nachteile aufgelistet, die durch interpretierte Sprachen oder
Skriptingsprachen entstehen und gleichzeitig wird ein Ausblick gegeben, was
dies für den eingebetteten Kontext bzw. Realzeitbetrieb bedeutet.


\begin{itemize}

  \item \textbf{Overhead}

        Es kann eine langsamere Ausführungszeit bei interpretierten Sprachen
        entstehen durch das höhere Abstraktionslevel und Vereinfachungen.
        Der Programmcode muss zur Laufzeit interpretiert bzw. compiliert
        werden; in regelmäßigen Abständen wird eine Garbage Collection
        vorgenommen, wodurch der eigentliche Programmablauf unterbrochen oder
        verlangsamt wird.

        Abgesehen von dem zusätzlichen Rechenaufwand gibt es einen Mehrbedarf
        an Arbeitsspeicher, zum einen benötigt der Interpreter (bzw. die
        virtuelle Maschine) zusätzlichen RAM für seinen Programmcode und seine
        Daten, zum anderen braucht der interpretierte Code ebenfalls mehr
        Speicher als beispielsweise C-Code.

        Die Ausführung mittels des Interpreters führt zu einer indirekteren
        Systeminteraktion, die Zugriffe auf das Betriebssystem (“Syscalls”)
        sind mit Overhead verbunden, da die Schnittstellen abstrahiert sind
        und der Interpreter die Aufrufe weiterleiten muss.

  \item \textbf{dynamische Typisierung}

        Die dynamische Typisierung kann allerdings auch ein Nachteil sein,
        da der Typ der Variablen unbekannt ist, was gerade in selten
        durchlaufenen Programmteilen, die eventuell auch nicht richtig
        getestet wurden, zu unerwartetem Fehlern führen kann. Die dynamische
        Typüberprüfung kostet natürlich auch Rechenaufwand zur Laufzeit,
        zudem fallen Optimierungsmöglichkeiten wie das direkte Einfügen von
        Maschinencode statt eines Methoden- oder Funktionsaufrufs weg.

\end{itemize}


\subsection{Problematiken im Realzeit-Bereich}


So lange das System noch nicht an der Auslastungsgrenze ist (CPU-Rechenzeit),
sollte es kein Problem darstellen, wenn die Software etwas langsamer läuft.
Es kommt erst dann zu Problemen, wenn harte Echtzeit von Nöten ist, wenn der
Garbage Collector anspringt und das Programm somit ins Stocken gerät -- zumal
der GC nicht unbedingt zu deterministischen Zeiten anspringt, und auch nicht
bekannt ist, wie lange er aktiv ist. Es ist somit sehr schwierig ein
Wort-Case-Szenario zu erstellen und es müssen Möglichkeiten gesucht werden,
wie man diese Problematiken in den Griff bekommen kann, so dass auch harte
Realzeit realisiert werden kann.


Für Systeme die nur eine weiche Realzeit erfordern, sollten diese
Beschränkungen eher weniger das Problem sein. Was z.B. bei diversen 
"`echtzeit"' Web-Anwendungen der Fall ist.

\section{Beispiel Python}


Python, dessen Name sich von der Künstlergruppe Monty Python ableitet, ist um
1990 von dem Niederländer Guido von Rossum entwickelt worden. In dieser
Sprache wird auf die klare Sprachsyntax und gute Lesbarkeit sehr großen Wert
gelegt. Geschaffen wurde die Sprache als Brücke zwischen Shell-Skripten und
C-Programmen.

Es handelt sich dabei um eine interpretierte, interaktive, Objekt-orientierte
und funktionale Sprache, die aber auch ein einfaches Skript sein kann. Es gibt
klar definierte Konzepte, wie Module für Namenräume, dynamische Typesierung
und simple, zugleich mächtige High-Level-Datenstrukturen wie Listen und
Verzeichnisse (Dictionaries.) Zudem ist Python sehr leicht via C/C++
erweiterbar, somit können Wrapper um C-Programme gebaut werden, so dass ein
erleichterter und komfortabler Zugriff via Python möglich wird. So können bei
Bedarf Programmzeilen in hart optimierten C-Code ausgelagert werden und Python
genutzt werden -- gerade im eingebetteten Kontext durchaus ein Vorteil.

Durch die aktive Entwicklergemeinde hat sich eine sehr große und mächtige
Standard-Bibliothek entwickelt. [pyref → library.pdf] Was nicht zu
unterschätzen ist, gerade weil dadurch die schon sehr portablen
Python-Programme, auf diese Basis zurückgreifen können, die ohne Konfiguration
auf anderen Systemen vornehmen zu müssen.

Aber Python ist auch sehr leicht über externe Bibliotheken erweiterbar.
Zudem gibt es neben der großen Standard-Bibliothek sehr viele Pakete, die via
pipy-Index zur Verfügung stehen -- dieser Umfasst zur Zeit 21658 Pakete.
[Fußnote auf pypi] Das Python-Ecosystem ist sehr umfassend!


\subsection{Architektur}


\subsubsection{Interpreter}


Die Python-Referenzimplementierung CPython ist in der Programmiersprache C
geschrieben. Ein Python-Skript wird über den CPython-Interpreter ausgeführt;
der Interpreter stellt alle Funktionalität bereit um das Skript “just in time”
auszuführen.

Module sind einfache *.py Dateien, die mit dem Schlüsselwort import als
Namensraum importiert werden können. Es kann dann auf Klassen und Funktionen
zugegriffen werden, die sich innerhalb der Datei befinden. Pakete sind
Ordner die eine \_\_init\_\_.py und auch wieder Module enthalten. Die Pakete
bzw. Module werden vom Interpreter relativ zu seinem Ort wo er von z.B. einer
Shell aufgerufen wurde aufgespürt. Der Interpreter spürt aber Pakete/Module
auch über die Umgebungsvariable PYTHONPATH auf oder über fest definierte
Ordner wo sich systemweite Pakete, wie die Standardbibliothek befinden.
Der Interpreter unterstützt eine selbständige Speicherverwaltung; eine
ausführliche Erklärung findet sich in [Kapitel x].

In CPython gibt es den General Interpreter Lock (GIL), welcher die CPython
Implementierung vereinfacht, indem er nur einen Thread zur gleichen Zeit
Python-Bytecode ausführen lässt. Threading ist durchaus möglich, um z.B.
Nicht-Blockierende Programmabschnitte zu ermöglichen -- aber mehrere CPU-Kerne
bringen der Anwendung nicht mehr Performanz. [pyref → reference.pdf S.81]


\subsubsection{Syntax}


Um ein Gefühl zu bekommen, wie Python-Code aussieht, ist hier ein Code-Beispiel
gezeigt, indem die meist verwandten Konstruktionen aufgezeigt werden.


\lstinputlisting[language=Python]{code/pycharm.py}


Es gibt noch eine Besonderheit bei den Python-Objekten: Dort muss explizit der
self-Zeiger bei jeder Funktion explizit mit angegeben werden. Was der Python
Philosophie  “Explicit is better than implicit” [pyzen] entspricht.


\subsubsection{Dokumentation}


Die Dokumentation in Python erfolgt über so genannte Docstrings. Diese
Docstrings werden an die Funktions-, Klassenrümpfe bzw. Modulrümpfe gehängt
(siehe Syntax-Beispiel.) Die Docstrings können während der Laufzeit mit
`funktionsname.\_\_doc\_\_` ausgelesen werden. Mit dem beliebten
Dokumentationsgenerator `sphinx` kann aus diesen Docstrings automatisch
eine Dokumentation generiert werden.
Beispiele, wie die Funktion benutzt wird, werden gerne als Doctests innerhalb
der Docstrings angegeben.


\addcontentsline{toc}{section}{Literaturverzeichnis}
\bibliographystyle{gerabbrv}
\bibliography{bib/references}

\end{document}


